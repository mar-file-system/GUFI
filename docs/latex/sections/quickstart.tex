% This file is part of GUFI, which is part of MarFS, which is released
% under the BSD license.
%
%
% Copyright (c) 2017, Los Alamos National Security (LANS), LLC
% All rights reserved.
%
% Redistribution and use in source and binary forms, with or without modification,
% are permitted provided that the following conditions are met:
%
% 1. Redistributions of source code must retain the above copyright notice, this
% list of conditions and the following disclaimer.
%
% 2. Redistributions in binary form must reproduce the above copyright notice,
% this list of conditions and the following disclaimer in the documentation and/or
% other materials provided with the distribution.
%
% 3. Neither the name of the copyright holder nor the names of its contributors
% may be used to endorse or promote products derived from this software without
% specific prior written permission.
%
% THIS SOFTWARE IS PROVIDED BY THE COPYRIGHT HOLDERS AND CONTRIBUTORS "AS IS" AND
% ANY EXPRESS OR IMPLIED WARRANTIES, INCLUDING, BUT NOT LIMITED TO, THE IMPLIED
% WARRANTIES OF MERCHANTABILITY AND FITNESS FOR A PARTICULAR PURPOSE ARE DISCLAIMED.
% IN NO EVENT SHALL THE COPYRIGHT HOLDER OR CONTRIBUTORS BE LIABLE FOR ANY DIRECT,
% INDIRECT, INCIDENTAL, SPECIAL, EXEMPLARY, OR CONSEQUENTIAL DAMAGES (INCLUDING,
% BUT NOT LIMITED TO, PROCUREMENT OF SUBSTITUTE GOODS OR SERVICES; LOSS OF USE,
% DATA, OR PROFITS; OR BUSINESS INTERRUPTION) HOWEVER CAUSED AND ON ANY THEORY OF
% LIABILITY, WHETHER IN CONTRACT, STRICT LIABILITY, OR TORT (INCLUDING NEGLIGENCE
% OR OTHERWISE) ARISING IN ANY WAY OUT OF THE USE OF THIS SOFTWARE, EVEN IF
% ADVISED OF THE POSSIBILITY OF SUCH DAMAGE.
%
%
% From Los Alamos National Security, LLC:
% LA-CC-15-039
%
% Copyright (c) 2017, Los Alamos National Security, LLC All rights reserved.
% Copyright 2017. Los Alamos National Security, LLC. This software was produced
% under U.S. Government contract DE-AC52-06NA25396 for Los Alamos National
% Laboratory (LANL), which is operated by Los Alamos National Security, LLC for
% the U.S. Department of Energy. The U.S. Government has rights to use,
% reproduce, and distribute this software.  NEITHER THE GOVERNMENT NOR LOS
% ALAMOS NATIONAL SECURITY, LLC MAKES ANY WARRANTY, EXPRESS OR IMPLIED, OR
% ASSUMES ANY LIABILITY FOR THE USE OF THIS SOFTWARE.  If software is
% modified to produce derivative works, such modified software should be
% clearly marked, so as not to confuse it with the version available from
% LANL.
%
% THIS SOFTWARE IS PROVIDED BY LOS ALAMOS NATIONAL SECURITY, LLC AND CONTRIBUTORS
% "AS IS" AND ANY EXPRESS OR IMPLIED WARRANTIES, INCLUDING, BUT NOT LIMITED TO,
% THE IMPLIED WARRANTIES OF MERCHANTABILITY AND FITNESS FOR A PARTICULAR PURPOSE
% ARE DISCLAIMED. IN NO EVENT SHALL LOS ALAMOS NATIONAL SECURITY, LLC OR
% CONTRIBUTORS BE LIABLE FOR ANY DIRECT, INDIRECT, INCIDENTAL, SPECIAL,
% EXEMPLARY, OR CONSEQUENTIAL DAMAGES (INCLUDING, BUT NOT LIMITED TO, PROCUREMENT
% OF SUBSTITUTE GOODS OR SERVICES; LOSS OF USE, DATA, OR PROFITS; OR BUSINESS
% INTERRUPTION) HOWEVER CAUSED AND ON ANY THEORY OF LIABILITY, WHETHER IN
% CONTRACT, STRICT LIABILITY, OR TORT (INCLUDING NEGLIGENCE OR OTHERWISE) ARISING
% IN ANY WAY OUT OF THE USE OF THIS SOFTWARE, EVEN IF ADVISED OF THE POSSIBILITY
% OF SUCH DAMAGE.



\section{Quickstart}
This guide assumes that the reader has some awareness of the purpose
of GUFI, but does not know how to set it up or use it. This guide
provides a full run through of the basics of building, installing,
setting up, and running GUFI within a short period of time.

What this guide does not do is provide information on every single
executable, flag, design decision, and implementation detail. For that,
see the other documentation.

\section{Download}
Obtain a copy of the source code from GitHub: \\

\indent \texttt{git clone}
\href{https://github.com/mar-file-system/GUFI.git}{\texttt{https://github.com/mar-file-system/GUFI.git}}

\section{Dependencies}
GUFI is known to build on Linux, macOS, and Windows (via cygwin).
Most of the dependencies that GUFI requires are part of standard
packages, such as \texttt{coreutils}, a C compiler, CMake, and
bash. However, even these dependencies on different systems are known
under different names. In \texttt{contrib/CI}, there are a series of
shell scripts that can be run on the systems they are named for to
download dependencies. Windows dependencies are listed in the GitHub
Actions YAML file located in \texttt{.github/workflows/test.yml}.

\section{Build and Install}
\subsection{Build Directory}
Create a directory to configure and build GUFI. The recommended build
directory name is \texttt{build} under the source tree root. It can be
anywhere that is not the source tree root itself. \\

\texttt{mkdir -p build \&\& cd build}

\subsection{Configure Build}
GUFI uses CMake to generate Makefiles. The method of running CMake
recommended here does not use the extensive list of flags available in
\texttt{cmake(1)}: \\

\texttt{cmake .. <flags>} \\

GUFI defines many CMake flags. Run \texttt{ccmake ..} to see the latest flags.
Some particularly useful ones are listed here: \\
\begin{table}[H]
  \centering
  \begin{tabularx}{\textwidth}{| l | X |}
    \hline
    CMake Variable & Explaination \\
    (prepend with \texttt{-D} when used) & \\
    \hline
    \texttt{CMAKE\_INSTALL\_PREFIX=<directory path>} & Where to install
    GUFI when runnng \texttt{make install}. Useful if not installing to
    system directories. \\
    \hline
    \texttt{DEP\_AI=<ON|OFF>} & Enable or disable building sqlite-vec,
    sqlite-lembed, llama.cpp, and other AI related files. \\
    & Note that llama.cpp is not bundled with GUFI, and is downloaded
    during the dependency build step. \\
    \hline
    \texttt{DEP\_BUILD\_THREADS=<thread count>} & Build dependencies with
    this many threads to speed up builds. \\
    \hline
    \texttt{DEP\_INSTALL\_PREFIX=<directory path>} & Place compiled
    bundled dependencies here. Used to keep the dependencies when the
    build directory is deleted. \\
    \hline
    \texttt{SERVER\_CONFIG=<file path>} & Name of file used to
    configure Python scripts on the server side (explained later). A
    copy of \texttt{server.example} will be placed into the parent
    directory of the provided path. Copy/rename this file to the
    provided configuration file name. This is the GUFI configuration
    file that is used in this guide. There is an additional
    configuration file that is not necessary and not described
    here. \\
    \hline
  \end{tabularx}
\end{table}

Configuring CMake on macOS is trickier than building on Linux,
especially with \texttt{DEP\_AI=On}. See the latest macOS CMake
configuration in \texttt{.github/workflows/test.yml}.

Configuring CMake on cygwin has a few caveats as well. See the latest
cygwin CMake configuration in \texttt{.github/workflows/test.yml}.

\subsection{Build}
\indent \indent \texttt{make}

\subsection{Install}
\indent \indent \texttt{(sudo) make install} \\

Due to how GUFI was planned to be deployed, this installs what is
known as the server side code. The server contains all of the actual
implementations of GUFI functionality. The implied client side
code does exist, but is not explained here.

\section {Usage}
With GUFI installed, usage is grouped into 2 phases. The first,
indexing, is normally only done by an administrator. Querying can be
done by both administrators and users, and has been split into 2
subsections (there is a third way, but has been intentionally left out
here because it is not necessary to get GUFI running).

\subsection{Phase 1: Index a Filesystem Tree}
There are 4 executables that can be used to index a filesystem
tree. Only \gufidirindex is described here. The common inputs are \\

\indent \gufidirindex \texttt{--threads <n> --index-xattrs tree GUFI\_tree\_parent} \\

where \texttt{tree} is the root of a filesystem tree to be
indexed. The basename of the directory path will be used to create the
index. \texttt{GUFI\_tree\_parent} is the directory above where the
index for this tree should be found. For example, passing in
\texttt{tree=a/b/c} and \texttt{GUFI\_tree\_parent=search} will result
in the index being generated in \texttt{search/c}. \texttt{GUFI\_tree\_parent}
was designed this way so that all calls to \gufidirindex can point to
a unified index location.

\subsubsection{Post-Indexing Optimizations}
There are a few operations that can be done after indexing a tree to
help increase search performance. These operations are entirely
optional. The executables used to do them are listed here, but are not
elaborated upon: \gufitreesummary, \gufitreesummaryall, and
\gufirollup.

\subsection{Phase 2: Querying}
\subsubsection{\gufiquery}
The primary command line tool for querying a GUFI tree is
\gufiquery. There are others, but this is the primary one to know
about.

Using \gufiquery requires knowledge of the internals of GUFI (such as
the schema of GUFI databases), and thus only ``advanced'' users are
expected to use it. The 3 primary flags used for passing in SQL,
\texttt{-T}, \texttt{-S}, and \texttt{-E}, correspond to the
tables/views that should be queried by them: the (optional)
\treesummary table, the (\texttt{directory})\summary table, and the
\entries table. Because the \treesummary table is optional, it will
not be elaborated on here. The (\texttt{directory})\summary table
contains \texttt{stat(2)} information about the directory, as well as
a series of columns that summarize the current directory, such as
total number of files and the total size of all files. The \entries
table contains \texttt{stat(2)} information on the individual files
and links.

There are several things to note that affect how \gufiquery should be
called (explainations can be found in the other latex documentation).

\begin{itemize}
\item The \entries table should not be used directly; instead, use the
  \vrpentries view.
\item The \summary table should not be used directly; instead, use the
  \vrsummary view.
\item \texttt{-S} is run before \texttt{-E}, and will cause
  \texttt{-E} to not run if \texttt{-S} does not output at least one
  row of results (say, due to a \texttt{WHERE} clause).
\item Nothing stops a user from querying an incorrect table from a
  flag e.g. querying \vrpentries from \texttt{-S}.
\item Many \href{https://sqlite.org/appfunc.html}{SQLite3 User Defined
  Functions} have been defined for GUFI. The one to know is
  \texttt{rpath(sname, sroll, [entry name])} (not usable with \summary
  and \entries/\pentries).
\end{itemize}

A simple \gufiquery call might looks something like: \\

\gufiquery \texttt{--threads <n>} \textbackslash \\
\indent \indent \texttt{-S "SELECT ... FROM vrsummary ... ;"} \textbackslash \\
\indent \indent \texttt{-E "SELECT ... FROM vrpentries ... ;"} \textbackslash \\
\indent \indent \texttt{GUFI\_tree} \\

where \texttt{GUFI\_tree} can be any directory generated by indexing. \\

A basic \texttt{find(1)} call such as \\

\texttt{find a/b/c} \\

can be replicated in GUFI like so: \\

\gufiquery \texttt{--threads <n>} \textbackslash \\
\indent \indent \texttt{-S "SELECT rpath(sname, sroll) FROM vrsummary;"} \textbackslash \\
\indent \indent \texttt{-E "SELECT rpath(sname, sroll, name) FROM vrpentries;"} \textbackslash \\
\indent \indent \texttt{search/c}

\paragraph{\gufivt}

\gufivt is a \href{https://sqlite.org/loadext.html}{SQLite3 Run-Time
  Loadable Extension} that wraps \gufiquery and makes it accessible via
the SQlite3 command line as well as via any language's SQLite3
library.

\subsubsection{User Facing Scripts}
Because \gufiquery is very complex and requires detailed knowledge of
the internals of GUFI, it is not the expected way most people will use
to query indexes. Instead, there is a series of Python scripts that
wrap \gufiquery (or one of the other binaries) to partially replicate
common filesystem search functionality with GUFI. The tools are:
\gufidu, \gufifind, \gufigetfattr, \gufils, and \gufistat.

Before running these scripts, the server configuration must be set
up. Look in the \texttt{SERVER\_CONFIG} parent directory. There should
be a file called \texttt{server.example}. Copy this file to
\texttt{SERVER\_CONFIG} and modify any values that need changing. At
minimum, \texttt{IndexRoot} will likely need to be changed to point to
\texttt{GUFI\_tree\_parent}\footnote{These need to be renamed to the
same variable name at some point.} or a subdirectory. An absolute path
is recommended.

With the configutation file set up, the user facing tools will
prefix every path passed to them with \texttt{IndexRoot}: \\

\texttt{In the configuration file:} \\
\indent \texttt{IndexRoot=search} \\
\indent ... \# other configuration values \\

\noindent Then, running \\

\indent \gufifind \texttt{c/d} \hfill \# \texttt{d} is some path under
\texttt{c} \\

will return all paths accessible by the caller under
\texttt{search/c/d}. Note that \gufifind has the same quirk as
\texttt{find(1)} where the input path(s) must come first in the
argument list. However, flag ordering does not matter. \\

Similarly, \\

\gufils \texttt{-al c/d} \\

will print all directories, files, and links immediately under
\texttt{search/c/d} formatted similarly to \texttt{ls(1) -al}.
