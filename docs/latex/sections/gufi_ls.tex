\section{gufi\_ls}
gufi\_ls functions like the POISX command ls. As with gufi\_find, there are a multitude of options available listed below

\subsection{Flags and Functionality}
\begin{table} [h]
\centering
\begin{tabular}{l|r}
Flags & Functionality\\\hline
--help & displays help menu\\
-v, --version & show program's version number and exit \\
-a, --all & do not ignore entries ending with .\\
-A, --almost-all & do not list implied . and .. \\
--block-size \textless block\_size\textgreater & with -l, scale sizes by block\_size when printing them \\
-B, --ignore-backups & do not list implied entries ending with ~\\
-G, --no-group & in a long listing, don't print group names\\
-i, --inode & print the index number of each file\\
-l & used a long listing format\\
-r, --reverse & reverse order while sorting\\
-R, --recursive & list sub-directories recursively\\
-s, --size & print the allocated size of each file in blocks\\
-S & sort by file size, largest first \\
--time-style \textless TIME\_STYLE\textgreater & time/date format with -l\\
-t & sort by modification time, newest first\\
-U & do not sort; list entries in directory order\\
--delim \textless c\textgreater & delimeter separating output columns\\ 
--in-memory-name \textless name\textgreater & Name of in-memory database when -R is used\\
--nlink-width \textless chars\textgreater & Width of nlink column\\
--size-width \textless chars\textgreater & Width of size column\\
--user-width \textless chars\textgreater & Width of user column\\
--group-width \textless chars\textgreater & width of group column
\end{tabular}
\caption{\label{fig:gufi_ls flags} {Flags and Functionality}}
\end{table}
