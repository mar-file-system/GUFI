% This file is part of GUFI, which is part of MarFS, which is released
% under the BSD license.
%
%
% Copyright (c) 2017, Los Alamos National Security (LANS), LLC
% All rights reserved.
%
% Redistribution and use in source and binary forms, with or without modification,
% are permitted provided that the following conditions are met:
%
% 1. Redistributions of source code must retain the above copyright notice, this
% list of conditions and the following disclaimer.
%
% 2. Redistributions in binary form must reproduce the above copyright notice,
% this list of conditions and the following disclaimer in the documentation and/or
% other materials provided with the distribution.
%
% 3. Neither the name of the copyright holder nor the names of its contributors
% may be used to endorse or promote products derived from this software without
% specific prior written permission.
%
% THIS SOFTWARE IS PROVIDED BY THE COPYRIGHT HOLDERS AND CONTRIBUTORS "AS IS" AND
% ANY EXPRESS OR IMPLIED WARRANTIES, INCLUDING, BUT NOT LIMITED TO, THE IMPLIED
% WARRANTIES OF MERCHANTABILITY AND FITNESS FOR A PARTICULAR PURPOSE ARE DISCLAIMED.
% IN NO EVENT SHALL THE COPYRIGHT HOLDER OR CONTRIBUTORS BE LIABLE FOR ANY DIRECT,
% INDIRECT, INCIDENTAL, SPECIAL, EXEMPLARY, OR CONSEQUENTIAL DAMAGES (INCLUDING,
% BUT NOT LIMITED TO, PROCUREMENT OF SUBSTITUTE GOODS OR SERVICES; LOSS OF USE,
% DATA, OR PROFITS; OR BUSINESS INTERRUPTION) HOWEVER CAUSED AND ON ANY THEORY OF
% LIABILITY, WHETHER IN CONTRACT, STRICT LIABILITY, OR TORT (INCLUDING NEGLIGENCE
% OR OTHERWISE) ARISING IN ANY WAY OUT OF THE USE OF THIS SOFTWARE, EVEN IF
% ADVISED OF THE POSSIBILITY OF SUCH DAMAGE.
%
%
% From Los Alamos National Security, LLC:
% LA-CC-15-039
%
% Copyright (c) 2017, Los Alamos National Security, LLC All rights reserved.
% Copyright 2017. Los Alamos National Security, LLC. This software was produced
% under U.S. Government contract DE-AC52-06NA25396 for Los Alamos National
% Laboratory (LANL), which is operated by Los Alamos National Security, LLC for
% the U.S. Department of Energy. The U.S. Government has rights to use,
% reproduce, and distribute this software.  NEITHER THE GOVERNMENT NOR LOS
% ALAMOS NATIONAL SECURITY, LLC MAKES ANY WARRANTY, EXPRESS OR IMPLIED, OR
% ASSUMES ANY LIABILITY FOR THE USE OF THIS SOFTWARE.  If software is
% modified to produce derivative works, such modified software should be
% clearly marked, so as not to confuse it with the version available from
% LANL.
%
% THIS SOFTWARE IS PROVIDED BY LOS ALAMOS NATIONAL SECURITY, LLC AND CONTRIBUTORS
% "AS IS" AND ANY EXPRESS OR IMPLIED WARRANTIES, INCLUDING, BUT NOT LIMITED TO,
% THE IMPLIED WARRANTIES OF MERCHANTABILITY AND FITNESS FOR A PARTICULAR PURPOSE
% ARE DISCLAIMED. IN NO EVENT SHALL LOS ALAMOS NATIONAL SECURITY, LLC OR
% CONTRIBUTORS BE LIABLE FOR ANY DIRECT, INDIRECT, INCIDENTAL, SPECIAL,
% EXEMPLARY, OR CONSEQUENTIAL DAMAGES (INCLUDING, BUT NOT LIMITED TO, PROCUREMENT
% OF SUBSTITUTE GOODS OR SERVICES; LOSS OF USE, DATA, OR PROFITS; OR BUSINESS
% INTERRUPTION) HOWEVER CAUSED AND ON ANY THEORY OF LIABILITY, WHETHER IN
% CONTRACT, STRICT LIABILITY, OR TORT (INCLUDING NEGLIGENCE OR OTHERWISE) ARISING
% IN ANY WAY OUT OF THE USE OF THIS SOFTWARE, EVEN IF ADVISED OF THE POSSIBILITY
% OF SUCH DAMAGE.



\section{\gufifind}
\gufifind is a wrapper script for \gufiquery that attempts to recreate
a large portion of GNU \find 's functionality.

One major difference between them is how arguments are parsed. In
\find, expression order matters. In \gufifind, expression order does
not matter.

\subsection{Flags}
\begin{longtable}{|l|p{0.75\linewidth}|}
  \hline
  Flag & Functionality \\
  \hline
  -h, -{}-help & show this help message and exit \\
  \hline
  -{}-version, -v & show program's version number and exit \\
  \hline
  -maxdepth levels & Descend at most levels (a non-negative integer)
  levels of directories below the command line arguments. -maxdepth 0
  means only apply the tests and actions to the command line
  arguments. \\
  \hline
  -mindepth levels & Do not apply any tests or actions at levels less
  than levels (a non-negative integer). -mindepth 1 means process all
  files except the command line arguments. \\
  \hline
  -amin n & File was last accessed n minutes ago. \\
  \hline
  -anewer file & File was last accessed more recently than file was modified. \\
  \hline
  -atime n & File was last accessed n*24 hours ago. \\
  \hline
  -cmin n & File's status was last changed n minutes ago. \\
  \hline
  -cnewer file & File's status was last changed more recently than file was modified. \\
  \hline
  -ctime n & File's status was last changed n*24 hours ago. \\
  \hline
  -empty & File is empty and is either a regular file or a directory. \\
  \hline
  -executable & Matches files which are executable and directories
  which are searchable (in a file name resolution sense). \\
  \hline
  -false & File is false and is either a regular file or a directory. \\
  \hline
  -gid n & File's numeric group ID is n. \\
  \hline
  -group gname & File belongs to group gname (numeric group ID allowed). \\
  \hline
  -iname pattern & Like -name, but the match is case insensitive (uses
  regex, not glob). \\
  \hline
  -inum n & File has inode number n. It is normally easier to use the
  -samefile test instead. \\
  \hline
  -iregex pattern & Like -regex, but the match is case insensitive. \\
  \hline
  -links n & File has n links. \\
  \hline
  -lname pattern & File is a symbolic link whose contents match shell pattern. \\
  \hline
  -mmin n & File's data was last modified n minutes ago. \\
  \hline
  -mtime n & File's data was last modified n*24 hours ago. \\
  \hline
  -name pattern & Base of file name (the path with the leading
  directories removed) matches shell pattern. \\
  \hline
  -newer file & File was modified more recently than file. \\
  \hline
  -path pattern & File name matches shell pattern. \\
  \hline
  -readable & Matches files which are readable. \\
  \hline
  -regex pattern & File name matches regular expression pattern. \\
  \hline
  -samefile name & File refers to the same inode as name. \\
  \hline
  -size n & Note that if searching for files of size less
  than some value (e.g. -2k), must use equal sign (e.g. -size=-2k) so
  that the value is not interpreted incorrectly by argparse. \\
  \hline
  -true & Always true. \\
  \hline
  -type c & File is of type c. \\
  \hline
  -uid n & File's numeric user ID is n. \\
  \hline
  -user uname & File is owned by user uname (numeric user ID allowed). \\
  \hline
  -writable & Matches files which are writable. \\
  \hline
  -fprint file & Output result to file \\
  \hline
  -ls & List current file ls -dils format. \\
  \hline
  -print & Print the full name on the standard output, followed by a
  newline, similar to GNU find. \\
  \hline
  -print0 & Print the full name on the standard output, followed by a
  null character, similar to GNU find. \\
  \hline
  -printf format & Print format on the standard output, similar to GNU
  find. All format specifiers other than \texttt{ACFTYZ} have been
  implemented. \\
  \hline
  \caption{\label{fig:gufi_find_flags}{\gufifind Flags and Functionality}}
\end{longtable}

Additionally, \gufifind has some flags that \find does not have:

\begin{longtable}{|l|p{0.655\linewidth}|}
  \hline
  Flag & Functionality \\
  \hline
  -{}-num-results n & First n results. \\
  \hline
  -{}-smallest & Sort output by size, ascending. \\
  \hline
  -{}-largest & Sort output by size, descending. \\
  \hline
  -{}-compress & Try to reduce memory usage by compressing with \\
  & zlib (if available). \hfill \\
  \hline
  -{}-delim c & Delimiter separating output columns. \\
  \hline
  -{}-in-memory-name name & Name of in-memory database when
  aggregation \\
  & is performed. \\
  \hline
  -{}-aggregate-name name & Name of final database when aggregation \\
  & is performed. \hfill \\
  \hline
  -{}-skip-file filename & Name of file containing directory basenames \\
  & to skip. \hfill \\
  \hline
  -V, -{}-verbose & Show the \gufiquery being executed. \\
  \hline
  \caption{\label{fig:gufi_find_ext}{\gufifind Extensions}}
\end{longtable}
