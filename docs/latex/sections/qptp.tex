% This file is part of GUFI, which is part of MarFS, which is released
% under the BSD license.
%
%
% Copyright (c) 2017, Los Alamos National Security (LANS), LLC
% All rights reserved.
%
% Redistribution and use in source and binary forms, with or without modification,
% are permitted provided that the following conditions are met:
%
% 1. Redistributions of source code must retain the above copyright notice, this
% list of conditions and the following disclaimer.
%
% 2. Redistributions in binary form must reproduce the above copyright notice,
% this list of conditions and the following disclaimer in the documentation and/or
% other materials provided with the distribution.
%
% 3. Neither the name of the copyright holder nor the names of its contributors
% may be used to endorse or promote products derived from this software without
% specific prior written permission.
%
% THIS SOFTWARE IS PROVIDED BY THE COPYRIGHT HOLDERS AND CONTRIBUTORS "AS IS" AND
% ANY EXPRESS OR IMPLIED WARRANTIES, INCLUDING, BUT NOT LIMITED TO, THE IMPLIED
% WARRANTIES OF MERCHANTABILITY AND FITNESS FOR A PARTICULAR PURPOSE ARE DISCLAIMED.
% IN NO EVENT SHALL THE COPYRIGHT HOLDER OR CONTRIBUTORS BE LIABLE FOR ANY DIRECT,
% INDIRECT, INCIDENTAL, SPECIAL, EXEMPLARY, OR CONSEQUENTIAL DAMAGES (INCLUDING,
% BUT NOT LIMITED TO, PROCUREMENT OF SUBSTITUTE GOODS OR SERVICES; LOSS OF USE,
% DATA, OR PROFITS; OR BUSINESS INTERRUPTION) HOWEVER CAUSED AND ON ANY THEORY OF
% LIABILITY, WHETHER IN CONTRACT, STRICT LIABILITY, OR TORT (INCLUDING NEGLIGENCE
% OR OTHERWISE) ARISING IN ANY WAY OUT OF THE USE OF THIS SOFTWARE, EVEN IF
% ADVISED OF THE POSSIBILITY OF SUCH DAMAGE.
%
%
% From Los Alamos National Security, LLC:
% LA-CC-15-039
%
% Copyright (c) 2017, Los Alamos National Security, LLC All rights reserved.
% Copyright 2017. Los Alamos National Security, LLC. This software was produced
% under U.S. Government contract DE-AC52-06NA25396 for Los Alamos National
% Laboratory (LANL), which is operated by Los Alamos National Security, LLC for
% the U.S. Department of Energy. The U.S. Government has rights to use,
% reproduce, and distribute this software.  NEITHER THE GOVERNMENT NOR LOS
% ALAMOS NATIONAL SECURITY, LLC MAKES ANY WARRANTY, EXPRESS OR IMPLIED, OR
% ASSUMES ANY LIABILITY FOR THE USE OF THIS SOFTWARE.  If software is
% modified to produce derivative works, such modified software should be
% clearly marked, so as not to confuse it with the version available from
% LANL.
%
% THIS SOFTWARE IS PROVIDED BY LOS ALAMOS NATIONAL SECURITY, LLC AND CONTRIBUTORS
% "AS IS" AND ANY EXPRESS OR IMPLIED WARRANTIES, INCLUDING, BUT NOT LIMITED TO,
% THE IMPLIED WARRANTIES OF MERCHANTABILITY AND FITNESS FOR A PARTICULAR PURPOSE
% ARE DISCLAIMED. IN NO EVENT SHALL LOS ALAMOS NATIONAL SECURITY, LLC OR
% CONTRIBUTORS BE LIABLE FOR ANY DIRECT, INDIRECT, INCIDENTAL, SPECIAL,
% EXEMPLARY, OR CONSEQUENTIAL DAMAGES (INCLUDING, BUT NOT LIMITED TO, PROCUREMENT
% OF SUBSTITUTE GOODS OR SERVICES; LOSS OF USE, DATA, OR PROFITS; OR BUSINESS
% INTERRUPTION) HOWEVER CAUSED AND ON ANY THEORY OF LIABILITY, WHETHER IN
% CONTRACT, STRICT LIABILITY, OR TORT (INCLUDING NEGLIGENCE OR OTHERWISE) ARISING
% IN ANY WAY OUT OF THE USE OF THIS SOFTWARE, EVEN IF ADVISED OF THE POSSIBILITY
% OF SUCH DAMAGE.



\subsection{\qptp}
Thread pools are generally written with a single work thread (with a
single lock) from which to threads pull from. This does not work past
a handful of threads. GUFI can run on hundreds of threads, and thus
required a more performant thread pool. The solution to this was
\qptp.

\qptp is named so because originally, there was one work queue (and
lock) maintained for each thread in the thread pool. Now, there are
multiple work queues maintained for each thread.

\subsubsection{Adding Work}
By default, work is added to threads in a round robin fashion in order
to distribute work evenly and to attempt to prevent, or at least
reduce the amount of, contention experienced by any one work queue.

This function can be changed during initialization.

\subsubsection{Processing Queued Work}
Because work items are enqueued in parallel while work items are
processed, popping off work items one at a time results in one lock
per removal that might experience contention while the queue is being
modified. In \qptp, when a thread pops off work for processing, {\bf
 all} work items in the queue are removed, resulting in the removal
of multiple work items with only a single lock that might experience
contention. All work is processed before the thread returns to the
work queue to find more work.

There exists a second work queue, called the deferred work queue, that
is pushed to if the thread pool is initialized with a non-zero
\texttt{queue\_limit}. Work items are placed in the deferred work
queue when the normal work queue has more than \texttt{queue\_limit}
items enqueued. The deferred work queue is only processed if the work
queue is empty when the worker thread goes to look for more work. The
work queue may still be pushed to if it was recently moved for
processing (since the work queue now has as size of 0). This changes
the order that work is processed, allowing for work to drain, reducing
memory pressure, while still continuously
processing work when there is work present.

If a thread discovers that it does not have work items in either queue
but the thread pool still has outstanding work, it will search for
more work in other threads. If work items are found, the thread will
steal some of them, causing the stolen work items to experience less
latency between enqueuing and processing. Note that when work items
are stolen, the work queue is searched first, and if nothing is found,
the deferred queue is searched next, instead of the next thread's work
queue.

\subsubsection{Usage}
\begin{enumerate}
\item Create a thread pool:

  \texttt{QPTPool\_t *pool = QPTPool\_init(nthreads, args);}

  \texttt{nthreads} sets the number of threads in this thread
  pool.

  The \texttt{args} argument will be accessible by all threads that
  are run.

\item Setting Properties:

  \texttt{QPTPool\_init} is intentionally kept simple and uses default
  values for some features. These values may be modified using
  \texttt{QPTPool\_init\_with\_props} or \texttt{QPTPool\_set\_*}
  functions before \texttt{QPTPool\_start} is called.

  By default, \qptp will push new work items in a round robin
  fashion. This can be changed to a custom function with
  \\\texttt{QPTPool\_set\_next}. This function is set at the context
  level instead of at \texttt{QPTPool\_enqueue} in order to not
  require a branch to figure out whether or not the provided function
  pointer is valid.

  \texttt{QPTPool\_set\_queue\_limit} causes work items to be pushed
  into the deferred work queue as described above.

  \texttt{QPTPool\_set\_steal} sets the numerator and denominator of
  the multiplier used when work items are being stolen from other
  threads. For the first queue where \texttt{queue.size * numerator /
  denominator} results in at least 1 work item, that many work items
  will be taken from the front of the queue.

\item Getting Properties:

  Properties may be extracted from the context using the
  \texttt{QPTPool\_get\_*} functions.

\item Start the thread pool:

  \texttt{QPTPool\_start(pool);}

\item Add work:

  \texttt{QPTPool\_enqueue(pool, id, function, work);}

  The function passed into \texttt{QPTPool\_enqueue} must match the
  signature found in \texttt{QueuePerThreadPool.h}. The \texttt{work} argument
  will only be accessible to the thread processing this work.

  The thread that will receive the new work item is not \texttt{id}.
  Rather, \texttt{id} is treated as the source thread id and
  \texttt{threads[id]->next\_queue} will be where the new work item is
  enqueued.

\item Wait for all work to be completed (threads are joined):

  \texttt{QPTPool\_wait(pool);}

  This function exists to allow for the collection of statistics
  before the context is destroyed.

\item Destroy the pool context:

  \texttt{QPTPool\_destroy(pool);}
\end{enumerate}
