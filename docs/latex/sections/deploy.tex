% This file is part of GUFI, which is part of MarFS, which is released
% under the BSD license.
%
%
% Copyright (c) 2017, Los Alamos National Security (LANS), LLC
% All rights reserved.
%
% Redistribution and use in source and binary forms, with or without modification,
% are permitted provided that the following conditions are met:
%
% 1. Redistributions of source code must retain the above copyright notice, this
% list of conditions and the following disclaimer.
%
% 2. Redistributions in binary form must reproduce the above copyright notice,
% this list of conditions and the following disclaimer in the documentation and/or
% other materials provided with the distribution.
%
% 3. Neither the name of the copyright holder nor the names of its contributors
% may be used to endorse or promote products derived from this software without
% specific prior written permission.
%
% THIS SOFTWARE IS PROVIDED BY THE COPYRIGHT HOLDERS AND CONTRIBUTORS "AS IS" AND
% ANY EXPRESS OR IMPLIED WARRANTIES, INCLUDING, BUT NOT LIMITED TO, THE IMPLIED
% WARRANTIES OF MERCHANTABILITY AND FITNESS FOR A PARTICULAR PURPOSE ARE DISCLAIMED.
% IN NO EVENT SHALL THE COPYRIGHT HOLDER OR CONTRIBUTORS BE LIABLE FOR ANY DIRECT,
% INDIRECT, INCIDENTAL, SPECIAL, EXEMPLARY, OR CONSEQUENTIAL DAMAGES (INCLUDING,
% BUT NOT LIMITED TO, PROCUREMENT OF SUBSTITUTE GOODS OR SERVICES; LOSS OF USE,
% DATA, OR PROFITS; OR BUSINESS INTERRUPTION) HOWEVER CAUSED AND ON ANY THEORY OF
% LIABILITY, WHETHER IN CONTRACT, STRICT LIABILITY, OR TORT (INCLUDING NEGLIGENCE
% OR OTHERWISE) ARISING IN ANY WAY OUT OF THE USE OF THIS SOFTWARE, EVEN IF
% ADVISED OF THE POSSIBILITY OF SUCH DAMAGE.
%
%
% From Los Alamos National Security, LLC:
% LA-CC-15-039
%
% Copyright (c) 2017, Los Alamos National Security, LLC All rights reserved.
% Copyright 2017. Los Alamos National Security, LLC. This software was produced
% under U.S. Government contract DE-AC52-06NA25396 for Los Alamos National
% Laboratory (LANL), which is operated by Los Alamos National Security, LLC for
% the U.S. Department of Energy. The U.S. Government has rights to use,
% reproduce, and distribute this software.  NEITHER THE GOVERNMENT NOR LOS
% ALAMOS NATIONAL SECURITY, LLC MAKES ANY WARRANTY, EXPRESS OR IMPLIED, OR
% ASSUMES ANY LIABILITY FOR THE USE OF THIS SOFTWARE.  If software is
% modified to produce derivative works, such modified software should be
% clearly marked, so as not to confuse it with the version available from
% LANL.
%
% THIS SOFTWARE IS PROVIDED BY LOS ALAMOS NATIONAL SECURITY, LLC AND CONTRIBUTORS
% "AS IS" AND ANY EXPRESS OR IMPLIED WARRANTIES, INCLUDING, BUT NOT LIMITED TO,
% THE IMPLIED WARRANTIES OF MERCHANTABILITY AND FITNESS FOR A PARTICULAR PURPOSE
% ARE DISCLAIMED. IN NO EVENT SHALL LOS ALAMOS NATIONAL SECURITY, LLC OR
% CONTRIBUTORS BE LIABLE FOR ANY DIRECT, INDIRECT, INCIDENTAL, SPECIAL,
% EXEMPLARY, OR CONSEQUENTIAL DAMAGES (INCLUDING, BUT NOT LIMITED TO, PROCUREMENT
% OF SUBSTITUTE GOODS OR SERVICES; LOSS OF USE, DATA, OR PROFITS; OR BUSINESS
% INTERRUPTION) HOWEVER CAUSED AND ON ANY THEORY OF LIABILITY, WHETHER IN
% CONTRACT, STRICT LIABILITY, OR TORT (INCLUDING NEGLIGENCE OR OTHERWISE) ARISING
% IN ANY WAY OUT OF THE USE OF THIS SOFTWARE, EVEN IF ADVISED OF THE POSSIBILITY
% OF SUCH DAMAGE.



\section{Deployment}
\label{sec:deploy}
GUFI functionality as described so far has been local (source
filesystem and index on the same node). However, this is not the
expected way to deploy GUFI at scale. Users should not log into the
machine containing both the filesystems and indexes with an
interactive shell. Instead, indexes are to be placed on a dedicated
GUFI server (which may or may not be able to see the source
filesystems) and users will call client scripts from client nodes to
forward commands to the server via ssh.

There are multiple reasons why GUFI is being deployed this way. First,
indexes should not be colocated with the source filesystems in order
to not use the resources of the source filesystem nodes. Second,
filesystems may not be visible to each other even if the indexes
should be (the indexes can all be colocated with of one of the source
filesystems, but then we are back at the first issue). Finally, the
indexes are placed on a dedicated server instead of a front-end node
because front-end nodes tend to be underpowered and are intended to be
used as jumping-off points to more powerful nodes, but GUFI queries,
which will be running in parallel and may be complex, will need a lot
of resources.

\subsection{User and Index Setup}
In order to prevent users from seeing data they should not see, first
ensure that the uids and gids point to the same people across all
filesystems being indexed. This should not be an issue for filesystems
whose users have fixed uids and gids.

The \texttt{/etc/passwd} from each filesystem's source system should
be merged into the existing one on the server in order to set up
the users. The login shells of every user should be set to
\texttt{/usr/sbin/nologin} so that users cannot log in directly (see
\ref{sec:hostbasedauth} for letting users in). The \texttt{/etc/group}
from each filesystem's source system should be similarly merged into
the server's \texttt{/etc/group}.

Individual indexes can then be moved onto the server, likely with
\\\gufitraceindex. Multiple indexes can be combined by simply placing
them under a common location. For example, when combining the indexes
\texttt{projects} and \texttt{scratch}, they can be both placed under
\texttt{search}, and the tools can be pointed to \texttt{search} to
search both indexes at once, or \texttt{search/projects} or
\texttt{search/scratch} to search them individually. Note that
\texttt{search} should (but does not have to) have an empty
\texttt{db.db} (the tables exist but are all empty) file in it.

\subsection{RPMs}
The build should be (re)configured with \texttt{cmake -DCLIENT=On} if
it is not already enabled. \texttt{make~package} will produce two RPMs
instead of one. The server RPM will contain the actual GUFI
implementation. The client RPM will contain renamed and modified
\guficlient wrapper scripts. These RPMs should be installed on their
respective nodes.

\subsubsection{Wrapper Scripts}
\gufiquery provides the core GUFI functionality. However, it is not
expected to be used by users, as the syntax of \gufiquery is somewhat
unwieldy, and users are not expected to understand how GUFI
works. Wrapper scripts such as \gufifind and \gufils were created in
order to provide users (and administrators) with predetermined queries
that are used often and extract useful information (see the user guide
for more details). The server has the actual implementation of the
wrapper scripts while the client has multiple copies of \guficlient
renamed to the corresponding script's name.

\subsection{Runtime Configuration}
The wrapper scripts on both the server and client require a valid GUFI
configuration file to function correctly. The path to the
configuration file can be set at configure time with the
\texttt{CONFIG\_FILE} CMake variable (defaults to
\guficonfigfile). The configuration file will be readable by all users
but should only be writable by the administrators (664 permissions) in
order to prevent non-admin users from changing how GUFI runs.

GUFI configuration files are simple text files containing lines of
\texttt{<key>=<value>}. Duplicate configuration keys are overwritten
by the bottom-most line. Empty lines and lines starting with \# are
ignored. The server and client require different configuration
values. Example configuration files can be found in the
\texttt{config} directory of the GUFI source, as well as in the build
directory. The corresponding examples will also be installed with
GUFI.

\subsubsection{Server}
\begin{tabular}{| l | l | l |}
  \hline
  Key & Value Type & Purpose \\
  \hline
  Threads & Positive Integer & Number of threads to use \\
  & & when running \gufiquery. \\
  \hline
  Query & File Path & The absolute path of \gufiquery. \\
  \hline
  Sqlite3 & File Path & The absolute path of \gufisqlite. \\
  \hline
  Stat & File Path & The absolute path of \gufistatbin. \\
  \hline
  IndexRoot & Directory Path & The absolute path of a GUFI tree \\
  & & (or parent of multiple indexes). \\
  \hline
  OutputBuffer & Non-negative Integer & The size of each per-thread
  buffer \\
  & & used to buffer writes to stdout. \\
  & & Setting this too low will affect \\
  & & performance. Setting this too high \\
  & & will use too much memory. \\
  \hline
\end{tabular}

\subsubsection{Client}
\begin{tabular}{| l | l | l |}
  \hline
  Key & Type & Purpose \\
  \hline
  Server & URI & The URI of the server (IP address, URL, etc.) \\
  \hline
  Port & Non-negative integer & Port number \\
  \hline
\end{tabular}

\subsubsection{Colocating the Server and Client}
Installing both the server and client on the same node with either the
RPMs or \texttt{make~install} results in a bad setup where either the
server wrapper scripts overwrite the client scripts or vice versa and
should not be done.

The only time when it makes sense for both the server and client to be
on the same node at the same time is during testing. When testing, the
client wrapper scripts should be run out of the GUFI build
directory. There, the client wrapper scripts will be called
\texttt{gufi\_client\_<name>}, e.g.,
\texttt{gufi\_client\_ls}. Additionally, the server and client
configurations can be merged into the same file if both the server and
client read from the same configuration file path.

\subsection{SSH}
Once the server and client are set up, users may be allowed in via the
client scripts forwarding commands through ssh.

\subsubsection{Authentication}
\label{sec:hostbasedauth}
Normal ssh, using passwords or user keys (\texttt{\textasciitilde
  /.ssh/id\_rsa}), should be disabled on the server. Instead, ssh
should be done using host-based authentication
(\texttt{/etc/ssh/ssh\_host\_rsa\_key}).

\subsubsection{\gufijail}
\label{sec:gufi_jail}
A restricted shell should be set up on the server to prevent arbitrary
command execution.

\gufijail is a simple script that limits the commands users logging in
with ssh are able to run to only a subset of GUFI
commands. \texttt{/etc/ssh/sshd\_config} should be modified with the
following lines:

\begin{verbatim}
    Match Group gufi
        ForceCommand /path/to/gufi_jail
\end{verbatim}
