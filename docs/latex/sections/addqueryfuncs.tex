% This file is part of GUFI, which is part of MarFS, which is released
% under the BSD license.
%
%
% Copyright (c) 2017, Los Alamos National Security (LANS), LLC
% All rights reserved.
%
% Redistribution and use in source and binary forms, with or without modification,
% are permitted provided that the following conditions are met:
%
% 1. Redistributions of source code must retain the above copyright notice, this
% list of conditions and the following disclaimer.
%
% 2. Redistributions in binary form must reproduce the above copyright notice,
% this list of conditions and the following disclaimer in the documentation and/or
% other materials provided with the distribution.
%
% 3. Neither the name of the copyright holder nor the names of its contributors
% may be used to endorse or promote products derived from this software without
% specific prior written permission.
%
% THIS SOFTWARE IS PROVIDED BY THE COPYRIGHT HOLDERS AND CONTRIBUTORS "AS IS" AND
% ANY EXPRESS OR IMPLIED WARRANTIES, INCLUDING, BUT NOT LIMITED TO, THE IMPLIED
% WARRANTIES OF MERCHANTABILITY AND FITNESS FOR A PARTICULAR PURPOSE ARE DISCLAIMED.
% IN NO EVENT SHALL THE COPYRIGHT HOLDER OR CONTRIBUTORS BE LIABLE FOR ANY DIRECT,
% INDIRECT, INCIDENTAL, SPECIAL, EXEMPLARY, OR CONSEQUENTIAL DAMAGES (INCLUDING,
% BUT NOT LIMITED TO, PROCUREMENT OF SUBSTITUTE GOODS OR SERVICES; LOSS OF USE,
% DATA, OR PROFITS; OR BUSINESS INTERRUPTION) HOWEVER CAUSED AND ON ANY THEORY OF
% LIABILITY, WHETHER IN CONTRACT, STRICT LIABILITY, OR TORT (INCLUDING NEGLIGENCE
% OR OTHERWISE) ARISING IN ANY WAY OUT OF THE USE OF THIS SOFTWARE, EVEN IF
% ADVISED OF THE POSSIBILITY OF SUCH DAMAGE.
%
%
% From Los Alamos National Security, LLC:
% LA-CC-15-039
%
% Copyright (c) 2017, Los Alamos National Security, LLC All rights reserved.
% Copyright 2017. Los Alamos National Security, LLC. This software was produced
% under U.S. Government contract DE-AC52-06NA25396 for Los Alamos National
% Laboratory (LANL), which is operated by Los Alamos National Security, LLC for
% the U.S. Department of Energy. The U.S. Government has rights to use,
% reproduce, and distribute this software.  NEITHER THE GOVERNMENT NOR LOS
% ALAMOS NATIONAL SECURITY, LLC MAKES ANY WARRANTY, EXPRESS OR IMPLIED, OR
% ASSUMES ANY LIABILITY FOR THE USE OF THIS SOFTWARE.  If software is
% modified to produce derivative works, such modified software should be
% clearly marked, so as not to confuse it with the version available from
% LANL.
%
% THIS SOFTWARE IS PROVIDED BY LOS ALAMOS NATIONAL SECURITY, LLC AND CONTRIBUTORS
% "AS IS" AND ANY EXPRESS OR IMPLIED WARRANTIES, INCLUDING, BUT NOT LIMITED TO,
% THE IMPLIED WARRANTIES OF MERCHANTABILITY AND FITNESS FOR A PARTICULAR PURPOSE
% ARE DISCLAIMED. IN NO EVENT SHALL LOS ALAMOS NATIONAL SECURITY, LLC OR
% CONTRIBUTORS BE LIABLE FOR ANY DIRECT, INDIRECT, INCIDENTAL, SPECIAL,
% EXEMPLARY, OR CONSEQUENTIAL DAMAGES (INCLUDING, BUT NOT LIMITED TO, PROCUREMENT
% OF SUBSTITUTE GOODS OR SERVICES; LOSS OF USE, DATA, OR PROFITS; OR BUSINESS
% INTERRUPTION) HOWEVER CAUSED AND ON ANY THEORY OF LIABILITY, WHETHER IN
% CONTRACT, STRICT LIABILITY, OR TORT (INCLUDING NEGLIGENCE OR OTHERWISE) ARISING
% IN ANY WAY OUT OF THE USE OF THIS SOFTWARE, EVEN IF ADVISED OF THE POSSIBILITY
% OF SUCH DAMAGE.



\clearpage
\subsection{User Defined Functions}
Convenience \href{https://www.sqlite.org/appfunc.html}{user defined functions} are added into \sqlite by GUFI.

\subsubsection{Functions available in \gufiquery and \gufisqlite}

\begin{table}[H]
  \centering
  \caption{General Utility Functions}
  \begin{tabularx}{\textwidth}{| l | X |}
    \hline
    Function & Description \\
    \hline
    \texttt{uidtouser(uid)} & Converts a UID to a user name \\
    \hline
    \texttt{gidtogroup(gid)} & Converts a GID to a group name \\
    \hline
    \texttt{modetotxt(mode)} & Converts numerical permission bits to text \\
    \hline
    \texttt{strftime(format, timestamp)} & Replaces \href{https://www.sqlite.org/lang_datefunc.html}{SQLite~3's custom \texttt{strftime}} with \texttt{strftime(3)} \\
    \hline
    \texttt{blocksize(bytes, unit)} & Converts a size to a the number of blocks of size \texttt{unit} needed to store the first argument. \texttt{unit} is the combination of at least one integer and/or a prefix and a suffix: \\
                                    & \\
                                    & Prefix: \texttt{K}, \texttt{M}, \texttt{G}, \texttt{T}, \texttt{P}, \texttt{E}\\
                                    & Suffix (multiplier): \textit{no suffix} (1024), \texttt{B} (1000), \texttt{iB} (1024) \\
                                    & \\
                                    & Return value is a string. \\
                                    & \\
                                    & Note that this function was implemented to replicate \texttt{ls} output and is meant for use with \gufils only, so use with caution. \\
                                    & \\
                                    & Examples: \\
                                    & \begin{tabular}{| l | l |}
                                        \hline
                                        Call                    & Output \\
                                        \hline
                                        blocksize(1024, `1000') & 2 \\
                                        \hline
                                        blocksize(1024, `1024') & 1 \\
                                        \hline
                                        blocksize(1024, `K')    & 1K \\
                                        \hline
                                        blocksize(1024, `1K')   & 1 \\
                                        \hline
                                        blocksize(1024, `KB')   & 2KB \\
                                        \hline
                                        blocksize(1024, `1KB')  & 2 \\
                                        \hline
                                        blocksize(1024, `KiB')  & 1KiB \\
                                        \hline
                                        blocksize(1024, `1KiB') & 1 \\
                                        \hline
                                      \end{tabular} \\
                                    & \\
    \hline
    \texttt{human\_readable\_size(bytes)} & Converts a size to a human readable string \\
    \hline
  \end{tabularx}
\end{table}

\begin{table}[H]
  \centering
  \caption{Get results from running commands with \texttt{popen(3)}}
  \begin{tabularx}{\textwidth}{| l | X |}
    \hline
    Function & Description \\
    \hline
    \texttt{strop(cmd)}  & Return the first line as a string          \\
    \hline
    \texttt{intop(cmd)}  & The only output should be an integer       \\
    \hline
    \texttt{blobop(cmd)} & Return all of \texttt{stdout} as one value \\
    \hline
  \end{tabularx}
\end{table}

\begin{table}[H]
  \centering
  \caption{\label{tab:sqlmath} Math functions added to complement the built-in math functions.}
  \begin{tabularx}{\textwidth}{| l | X |}
    \hline
    Function & Description \\
    \hline
    \texttt{median(numeric column)} & Median of values \\
    \hline
    \texttt{mode\_count\_hist(value)} & See Table~\ref{tab:sqlhist} \\
    \hline
    \texttt{stdevp(numeric column)} & Population standard deviation \\
    \hline
    \texttt{stdevs(numeric column)} & Sample standard deviation \\
    \hline
  \end{tabularx}
\end{table}

\begin{table}[H]
  \centering
  \caption{\label{tab:sqlhist}Construct histograms within SQLite~3. These functions return formatted strings for parsing outside of SQLite~3. See \texttt{include/histogram.h} for details.}
  \begin{tabularx}{\textwidth}{| l | X | }
    \hline
    Function & Description \\
    \hline
    \texttt{log2\_hist(input, bucket\_count)} & Count \texttt{input} in the [2\textsuperscript{i},~2\textsuperscript{i + 1}) bucket, where $i = \lfloor \log_2 (input) \rfloor$ \\
    \hline
    \texttt{mode\_hist(permission bits)} & Count permission bit occurances. Buckets are 000~-~777 (octal). Input should be integers, not octal or human readable strings. \\
    \hline
    \texttt{time\_hist(timestamp, ref)} & Bucket \texttt{(ref - timestamp)} in human understandable durations \\
    \hline
    \texttt{category\_hist(value, keep\_1)} & Each unique value is its own bucket. \texttt{keep\_1} determines whether or not to keep categories with a count of 1 when returning the formatted string. \\
    \hline
    \texttt{category\_hist\_combine(string)} & Combine multiple category histograms into one \\
    \hline
    \texttt{mode\_count\_hist(value)} & Get the value with the highest occurance. If multiple values have the same highest occurance, return \texttt{NULL}. \\
    \hline
  \end{tabularx}
\end{table}

\subsubsection{Functions available only in \gufiquery}

\begin{table}[H]
  \centering
  \caption{These functions involve the thread state of \gufiquery}
  \begin{tabularx}{\textwidth}{| l | X |}
    \hline
    Function & Description \\
    \hline
    \texttt{setstr(key, value)} & Define a user variable to do SQL string replacement on in a later query. See \ref{sec:user_strings} for details. \\
    \hline
    \texttt{thread\_id()} & Thread ID of the thread processing the current directory \\
    \hline
  \end{tabularx}
\end{table}

\begin{table}[H]
  \centering
  \caption{\label{tab:sqlwcontext}Functions that require the context of an index}
  \begin{tabularx}{\textwidth}{| l | X |}
    \hline
    Function & Description \\
    \hline
    \texttt{path()} & Current directory relative to path passed into executable \\
    \hline
    \texttt{epath()} & Current directory basename \\
    \hline
    \texttt{fpath()} & Full path of current directory \\
    \hline
    \rpath(sname, sroll [, name]) & Current directory relative to path passed into executable
                                    taking into account the rolled up name in summary table
                                    and rollup score. Should only be used with the \texttt{sname}
                                    and \texttt{sroll} columns of the \vrpentries, \vrsummary,
                                    \vrxpentries, and \vrxsummary views. \\
                                  & \\
                                  & Usage: \\
                                  & \texttt{SELECT rpath(sname, sroll)} \\
                                  & \texttt{FROM} \vrsummary /\vrxsummary; \\
                                  & \\
                                  & \texttt{SELECT rpath(sname, sroll, name)} \\
                                  & \texttt{FROM} \vrpentries /\vrxpentries; \\
                                  & \\
                                  & which is equivalent to: \\
                                  & \\
                                  & \texttt{SELECT rpath(sname, sroll) || `/' || name} \\
                                  & \texttt{FROM} \vrpentries /\vrxpentries; \\
    \hline
    \texttt{level()} & Depth of the current directory from the starting directory \\
    \hline
    \texttt{starting\_point()} & Path of the starting directory \\
    \hline
    \texttt{subdirs(srollsubdirs, sroll)} & Number of subdirectories under a directory. Only
                                            available for use with \vrsummary. When the index
                                            is not rolled up, the value is retrieved from C.
                                            When the index is rolled up, the value is retrieved
                                            from \vrsummary. \\
    \hline
  \end{tabularx}
\end{table}

\clearpage
