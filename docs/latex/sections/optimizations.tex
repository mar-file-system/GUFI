% This file is part of GUFI, which is part of MarFS, which is released
% under the BSD license.
%
%
% Copyright (c) 2017, Los Alamos National Security (LANS), LLC
% All rights reserved.
%
% Redistribution and use in source and binary forms, with or without modification,
% are permitted provided that the following conditions are met:
%
% 1. Redistributions of source code must retain the above copyright notice, this
% list of conditions and the following disclaimer.
%
% 2. Redistributions in binary form must reproduce the above copyright notice,
% this list of conditions and the following disclaimer in the documentation and/or
% other materials provided with the distribution.
%
% 3. Neither the name of the copyright holder nor the names of its contributors
% may be used to endorse or promote products derived from this software without
% specific prior written permission.
%
% THIS SOFTWARE IS PROVIDED BY THE COPYRIGHT HOLDERS AND CONTRIBUTORS "AS IS" AND
% ANY EXPRESS OR IMPLIED WARRANTIES, INCLUDING, BUT NOT LIMITED TO, THE IMPLIED
% WARRANTIES OF MERCHANTABILITY AND FITNESS FOR A PARTICULAR PURPOSE ARE DISCLAIMED.
% IN NO EVENT SHALL THE COPYRIGHT HOLDER OR CONTRIBUTORS BE LIABLE FOR ANY DIRECT,
% INDIRECT, INCIDENTAL, SPECIAL, EXEMPLARY, OR CONSEQUENTIAL DAMAGES (INCLUDING,
% BUT NOT LIMITED TO, PROCUREMENT OF SUBSTITUTE GOODS OR SERVICES; LOSS OF USE,
% DATA, OR PROFITS; OR BUSINESS INTERRUPTION) HOWEVER CAUSED AND ON ANY THEORY OF
% LIABILITY, WHETHER IN CONTRACT, STRICT LIABILITY, OR TORT (INCLUDING NEGLIGENCE
% OR OTHERWISE) ARISING IN ANY WAY OUT OF THE USE OF THIS SOFTWARE, EVEN IF
% ADVISED OF THE POSSIBILITY OF SUCH DAMAGE.
%
%
% From Los Alamos National Security, LLC:
% LA-CC-15-039
%
% Copyright (c) 2017, Los Alamos National Security, LLC All rights reserved.
% Copyright 2017. Los Alamos National Security, LLC. This software was produced
% under U.S. Government contract DE-AC52-06NA25396 for Los Alamos National
% Laboratory (LANL), which is operated by Los Alamos National Security, LLC for
% the U.S. Department of Energy. The U.S. Government has rights to use,
% reproduce, and distribute this software.  NEITHER THE GOVERNMENT NOR LOS
% ALAMOS NATIONAL SECURITY, LLC MAKES ANY WARRANTY, EXPRESS OR IMPLIED, OR
% ASSUMES ANY LIABILITY FOR THE USE OF THIS SOFTWARE.  If software is
% modified to produce derivative works, such modified software should be
% clearly marked, so as not to confuse it with the version available from
% LANL.
%
% THIS SOFTWARE IS PROVIDED BY LOS ALAMOS NATIONAL SECURITY, LLC AND CONTRIBUTORS
% "AS IS" AND ANY EXPRESS OR IMPLIED WARRANTIES, INCLUDING, BUT NOT LIMITED TO,
% THE IMPLIED WARRANTIES OF MERCHANTABILITY AND FITNESS FOR A PARTICULAR PURPOSE
% ARE DISCLAIMED. IN NO EVENT SHALL LOS ALAMOS NATIONAL SECURITY, LLC OR
% CONTRIBUTORS BE LIABLE FOR ANY DIRECT, INDIRECT, INCIDENTAL, SPECIAL,
% EXEMPLARY, OR CONSEQUENTIAL DAMAGES (INCLUDING, BUT NOT LIMITED TO, PROCUREMENT
% OF SUBSTITUTE GOODS OR SERVICES; LOSS OF USE, DATA, OR PROFITS; OR BUSINESS
% INTERRUPTION) HOWEVER CAUSED AND ON ANY THEORY OF LIABILITY, WHETHER IN
% CONTRACT, STRICT LIABILITY, OR TORT (INCLUDING NEGLIGENCE OR OTHERWISE) ARISING
% IN ANY WAY OUT OF THE USE OF THIS SOFTWARE, EVEN IF ADVISED OF THE POSSIBILITY
% OF SUCH DAMAGE.



\subsection{Optimizations}
In order for GUFI to be performant, many optimizations were used and
implemented.

\subsubsection{Reduced Branching}
In order to reduce the number of failed branch predictions experienced
by GUFI, branching was removed where possible. The main way this was
done was by intentionally skipping \texttt{NULL} pointer checks that
are repeated or always expected to be valid.

\subsubsection{Allocations}
Allocations are not performed by the standard malloc. Instead,
\texttt{jemalloc(3)} is used to override \texttt{malloc(3)}. See
\href{https://jemalloc.net/}{jemalloc's website} for details.

\subsubsection{Not Calling \lstat During Tree Walk}
\texttt{struct~dirent}s are returned when reading a directory with
\readdir. glibc's implementation of \texttt{struct~dirent} provides
extra fields not required by POSIX.1. GUFI takes advantage of the
nonstandard \texttt{d\_type} field to not call \lstat when determining
whether or not the entry is a directory. Only when \texttt{d\_type} is
not set will \lstat be called.

\subsubsection{Enqueuing Work Before Processing}
In order to reduce the amount of time the thread pool spends waiting
for work, work is enqueued before doing the main processing for a
directory. When walking a tree, such as with \gufidirindex,
\gufidirtrace, and \gufiquery, subdirectories are enqueued before
processing source filesystem information and running
queries. \gufitraceindex reads each trace file using a single
thread. However, each stanza is not processed as it is
encountered. Instead, each stanza's file offset is enqueued in the
thread pool, allowing for the directories of the index to be generated
in parallel.

\subsubsection{String Manipulations}
Where possible, strings are not copied, and are instead referenced.
Additionally, calls to \texttt{strlen(3)} are avoided in order to not
walk memory one byte at a time. Some string lengths, such as those for
input arguments are obtained with \texttt{strlen(3)}. Afterwards,
string manipulations are done by using or modifying lengths to offset
into or cut short strings.

\subsubsection{Combining Strings with \memcpy}
One method of combining C-strings is by concantenating them with
\texttt{snprintf(3)} with format strings containing only \texttt{\%s}
format specifiers. Instead of parsing the format string, the
\texttt{SNFORMAT\_S} function was created to do \memcpy s on the
arguments, skipping figuring out whether or not inputs are strings and
how long they are by finding \texttt{NULL} terminators. Instead,
lengths are obtained as by-products of previous string manipulations
and the values are reused.

\subsubsection{In-Situ Processing}
Occassionally, GUFI might encounter extremely large directories. This
results in many long lived dynamic allocations being created during
descent, which can overwhelm memory. Users can set a subdirectory
limit so that if too many subdirectories are encountered within a
single directory, subdirectories past the user provided count will be
processed recursively using a work item allocated on the thread's
stack instead of being dynamically allocated and enqueued for
processing. This reduces memory pressure by limiting the amount of
work items that extremely large directories would otherwise spawn. The
subdirectories being processed recursively may themselves enqueue
dynamically allocated subdirectory work or recurse further down with
subdirectory work allocated on the stack.

\subsubsection{Smaller Enqueued Work Items}
The main data structure that is enqueued is \work. This
struct started at approximately 14KiB in size.

\href{https://github.com/mar-file-system/GUFI/commit/2227d00665eb6d507ac2052e80616c077a5da853}{2227d00}
moved the parts of this structure that were not necssary for directory
tree traversal to \texttt{struct~entry\_data}, reducing
\work to slightly over 8KiB. This was used to fix
\href{https://github.com/mar-file-system/GUFI/issues/121}{Issue~121}.

\href{https://github.com/mar-file-system/GUFI/commit/e260d26aab0713835defbe2a5b6e1187610dcf3d}{e260d26}
changed \work so that the name field was a flexible
array stored after the other struct members, reducing
\work down to $\sim$200~bytes (depends on the length of the
path). However, C++ does not support flexible arrays, so
\href{https://github.com/mar-file-system/GUFI/commit/d1265f9ee9a5aa4451400f5abb455eb3ad64957c}{d1265f9}
was pushed, returning the \texttt{name} member back to its original
position, but making it a pointer that points to a location after the
struct.

\subsubsection{Compression with zlib}
When zlib is detected during CMake configuration, \work can be
compressed to further reduce the size of each work item that is
sitting in memory waiting to be processed. The compressed buffer,
originally allocated with
\texttt{sizeof(struct~work)~+~name\_len~+~1)} bytes, is then
reallocated to the compressed size. The bulk of \work is made up of
text strings followed by \texttt{NULL} characters, both of which are
highly compressible, meaning that compressed work items can be
expected to be much smaller than uncompressed work items.

Note that \work is its own compressed buffer. Whether
or not the work item is compressed and the compressed length are now
the first two fields of \work. When a work item is
compressed, a pointer pointing to it will have less space allocated to
it than \texttt{sizeof(struct~work)~+~name\_len~+~1)}.

This was used to fix
\href{https://github.com/mar-file-system/GUFI/issues/121}{Issue 121}.

\subsubsection{Database Templates}
Every directory in an index contains at least one database file,
called db.db, containing the \lstat data from the source
filesystem. When creating indexes, a database file is created with
the same schema as db.db and is left unfilled. When each directory in
the index is processed, the database file created earlier is copied
into the directory as a bytestream instead of having SQlite open new
database files for each directory. This avoids the multitudes of
checks done by SQLite when setting up databases and tables. The same
is done for external xattr database files.

\subsubsection{SQLite}
As SQLite is a major component in GUFI, attempts were made to optimize
its usage. Some optimizations were made at compile time. See the
\href{https://www.sqlite.org/compile.html}{SQLite Compile-time
  Options} page for details.

\paragraph{Locking}
In order to prevent multiple threads from corrupting data, SQLite
implements locking. In GUFI, each database is only ever accessed by
one thread:

\begin{itemize}
\item When indexing, only one thread writes to each directory's
  database.
\item When querying, the per-thread results are written to per-thread
  databases. After the tree walk, the per-thread databases are merged
  serially into a final database.
\end{itemize}

Locking despite never modifying databases in parallel is not useful,
and was removed by setting \texttt{-DSQLITE\_THREADSAFE=0} in the
compile flags.

\paragraph{VFS}
In addition to not locking SQLite in-memory operations, locking at the
filesystem level was also disabled. Instead of opening SQLite database
files with the default VFS, GUFI uses the \texttt{unix-none} VFS,
which causes all file locking to become no-ops. See
\href{https://www.sqlite.org/vfs.html}{The SQLite OS Interface or
 ``VFS''} for details.

\paragraph{Memory Tracking}
Memory tracking was disabled with \\
\noindent \texttt{-DSQLITE\_DEFAULT\_MEMSTATUS=0}.

\paragraph{Temporary Files}
Temporary files can be stored to disk or in memory. GUFI forces all
temporary files to be stored in memory with \\
\texttt{-DSQLITE\_TEMP\_STORE=3}.

\subsubsection{Caching Queries (Not Merged)}
When queries are performed on indexes, they are processed from
scratch by each thread for each directory. An obvious optimization
would be to reduce the amount of string parsing and query planning by
compiling each query once (or a small number of times such as once per
thread) at the beginning of a run and saving the compiled queries for
repeated use during the index traversal.

An attempt at caching queries was made with
\href{https://github.com/mar-file-system/GUFI/pull/95}{Pull~Request~\#95}.
Unfortunately, caching queries at best seemed to perform on par with
the latest GUFI and at worst, slightly slower than the latest
GUFI. This was true for both simple queries and complex queries with
\texttt{JOIN}s.
