\section{gufi\_stats}
gufi\_stats is used to analyze a tree and return generic statistics the user requests for

\subsection{Stats list}
\begin{table} [h]
\centering
\begin{tabular}{l}
Stats\\\hline
depth \\
filesize \\
filecount \\
linkcount \\
dircount \\
leaf-dirs \\
leaf-depth \\
leaf-files \\
leaf-links \\
total-filesize \\
total-filecount \\
total-linkcount \\
total-dircount \\
total-leaf-files \\
total-leaf-links \\
files-per-level \\ 
links-per-level \\
dirs-per-level \\
average-leaf-files \\
average-leaf-links \\ 
median-leaf-files \\
duplicate-names
\end{tabular}
\caption{\label{fig:stat_options}List of available stats}
\end{table}

\begin{table} [h]
\centering
\begin{tabular}{l|r}
Optional Flags & Functionality\\\hline
--help & displays help menu\\ 
--version, -v & display program's version number and exits \\
--recursive, -r & run command recursively \\
--cumulative -c & return cumulative values \\
--order \textless order\textgreater & sort output (if applicable)\\
--delim \textless c\textgreater & delimeter separating output columns\\
--num-results \textless n\textgreater & first n results \\
--uid \textless u\textgreater, --user \textless u\textgreater & restrict to user \\
--in-memory-name \textless name\textgreater & Name of in-memory database when -R is used
\end{tabular}
\caption{\label{fig:gufi_stats_flags}Flags and Functionality}
\end{table}

\subsection{Structure and example call}
Structure: \texttt{gufi\_stats [optional flags] [stat] }
\\
\\
EX: \texttt{./gufi\_stats --delim + total-filecount}