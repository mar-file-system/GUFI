% This file is part of GUFI, which is part of MarFS, which is released
% under the BSD license.
%
%
% Copyright (c) 2017, Los Alamos National Security (LANS), LLC
% All rights reserved.
%
% Redistribution and use in source and binary forms, with or without modification,
% are permitted provided that the following conditions are met:
%
% 1. Redistributions of source code must retain the above copyright notice, this
% list of conditions and the following disclaimer.
%
% 2. Redistributions in binary form must reproduce the above copyright notice,
% this list of conditions and the following disclaimer in the documentation and/or
% other materials provided with the distribution.
%
% 3. Neither the name of the copyright holder nor the names of its contributors
% may be used to endorse or promote products derived from this software without
% specific prior written permission.
%
% THIS SOFTWARE IS PROVIDED BY THE COPYRIGHT HOLDERS AND CONTRIBUTORS "AS IS" AND
% ANY EXPRESS OR IMPLIED WARRANTIES, INCLUDING, BUT NOT LIMITED TO, THE IMPLIED
% WARRANTIES OF MERCHANTABILITY AND FITNESS FOR A PARTICULAR PURPOSE ARE DISCLAIMED.
% IN NO EVENT SHALL THE COPYRIGHT HOLDER OR CONTRIBUTORS BE LIABLE FOR ANY DIRECT,
% INDIRECT, INCIDENTAL, SPECIAL, EXEMPLARY, OR CONSEQUENTIAL DAMAGES (INCLUDING,
% BUT NOT LIMITED TO, PROCUREMENT OF SUBSTITUTE GOODS OR SERVICES; LOSS OF USE,
% DATA, OR PROFITS; OR BUSINESS INTERRUPTION) HOWEVER CAUSED AND ON ANY THEORY OF
% LIABILITY, WHETHER IN CONTRACT, STRICT LIABILITY, OR TORT (INCLUDING NEGLIGENCE
% OR OTHERWISE) ARISING IN ANY WAY OUT OF THE USE OF THIS SOFTWARE, EVEN IF
% ADVISED OF THE POSSIBILITY OF SUCH DAMAGE.
%
%
% From Los Alamos National Security, LLC:
% LA-CC-15-039
%
% Copyright (c) 2017, Los Alamos National Security, LLC All rights reserved.
% Copyright 2017. Los Alamos National Security, LLC. This software was produced
% under U.S. Government contract DE-AC52-06NA25396 for Los Alamos National
% Laboratory (LANL), which is operated by Los Alamos National Security, LLC for
% the U.S. Department of Energy. The U.S. Government has rights to use,
% reproduce, and distribute this software.  NEITHER THE GOVERNMENT NOR LOS
% ALAMOS NATIONAL SECURITY, LLC MAKES ANY WARRANTY, EXPRESS OR IMPLIED, OR
% ASSUMES ANY LIABILITY FOR THE USE OF THIS SOFTWARE.  If software is
% modified to produce derivative works, such modified software should be
% clearly marked, so as not to confuse it with the version available from
% LANL.
%
% THIS SOFTWARE IS PROVIDED BY LOS ALAMOS NATIONAL SECURITY, LLC AND CONTRIBUTORS
% "AS IS" AND ANY EXPRESS OR IMPLIED WARRANTIES, INCLUDING, BUT NOT LIMITED TO,
% THE IMPLIED WARRANTIES OF MERCHANTABILITY AND FITNESS FOR A PARTICULAR PURPOSE
% ARE DISCLAIMED. IN NO EVENT SHALL LOS ALAMOS NATIONAL SECURITY, LLC OR
% CONTRIBUTORS BE LIABLE FOR ANY DIRECT, INDIRECT, INCIDENTAL, SPECIAL,
% EXEMPLARY, OR CONSEQUENTIAL DAMAGES (INCLUDING, BUT NOT LIMITED TO, PROCUREMENT
% OF SUBSTITUTE GOODS OR SERVICES; LOSS OF USE, DATA, OR PROFITS; OR BUSINESS
% INTERRUPTION) HOWEVER CAUSED AND ON ANY THEORY OF LIABILITY, WHETHER IN
% CONTRACT, STRICT LIABILITY, OR TORT (INCLUDING NEGLIGENCE OR OTHERWISE) ARISING
% IN ANY WAY OUT OF THE USE OF THIS SOFTWARE, EVEN IF ADVISED OF THE POSSIBILITY
% OF SUCH DAMAGE.



\section{gufi\_query}

\gufiquery is the main tool used for accessing indicies. Arbitrary SQL
statements are passed into \gufiquery to run on the individual
databases, allowing for anything SQL can do to be done on the
databases in an index, including modifying the contents of each
database. To prevent accidental modifications from occuring, indicies
are opened in read-only mode by default.

Because \gufiquery uses SQL statements for database access, caller
will need to know about GUFI's database and table schemas. We expect
only ``advanced'' users such as administrators and developers to use
\gufiquery directly. Users who do not know how GUFI functions may call
\gufiquery, but might not be able to use it properly.

\subsection{Flags}

\begin{table} [H]
  \centering
  \begin{tabular*}{\linewidth}{l|p{0.5\linewidth}}
    Flag & Functionality \\\hline
    -h & help\\
    \hline
    -H & show assigned input values (debugging)\\
    \hline
    -E \textless SQL ent\textgreater & SQL for entries table \\
    \hline
    -S \textless SQL sum\textgreater (will be read first) & SQL for summary table\\
    \hline
    -T \textless SQL tsum\textgreater & SQL for tree-summary table\\
    \hline
    -a & AND/OR (SQL query combination)\\
    \hline
    -n \textless threads\textgreater & number of threads\\
    \hline
    -j & print the information in terse form\\
    \hline
    -o \textless out\_fname\textgreater & output file (one-per thread, with thread-id suffix) implies -e 1\\
    \hline
    -d \textless delim\textgreater & one char delimiter \\
    \hline
    -O \textless out\_DB\textgreater & output DB, implies -e 1 \\
    \hline
    -I \textless SQL\_init\textgreater & SQL init \\
    \hline
    -F \textless SQL\_fin\textgreater & SQL cleanup \\
    \hline
    -y \textless min-level\textgreater & minimum level to descend to \\
    \hline
    -z \textless max-level\textgreater & maximum level to descend to\\
    \hline
    -J \textless SQL\_interm\textgreater & SQL for intermediate results (no default: recommend using "SELECT * FROM entries") \\
    \hline
    -K \textless create aggregate\textgreater & SQL to create the final aggregation table (if not specified, -I will be used)\\
    \hline
    -G \textless SQL\_aggregate\textgreater & SQL for aggregated results (no default: recommend using "SELECT * FROM entries")\\
    \hline
    -e \textless 0 or 1\textgreater & 0 for aggregate, 1 for print without aggregating (implied by -o and -O)\\
    \hline
    -m & Keep mtime and atime same on the database files \\
    \hline
    -B \textless buffer size\textgreater & size of each thread's output buffer in bytes \\
    \hline
    -w & open the database files in read-write mode instead of read only mode
  \end{tabular*}
  \caption{\label{tab:widgets} \gufiquery Flags and Arguments}
\end{table}

\subsection{Example Calls}

\noindent \gufiquery \texttt{-S "SELECT * FROM summary"
  ~/directory\_of\_root\_index}

\noindent \gufiquery \texttt{-E "SELECT * FROM pentries"
  ~/directory\_of\_root\_index}

\subsection{Visualizing the Workflow}
\begin{figure} [H]
  \centering
  \includegraphics[width=\textwidth]{images/gufi_query_main.png}
  \caption{Workflow of \gufiquery}
\end{figure}

\subsection{\processdir}
The core of \gufiquery is the \processdir function. This is where the
-T, -S, and -E flags are processed. Multiple instances of this
function are run in parallel via the thread pool in order to quickly
traverse and process an index.

\begin{figure} [H]
  \centering
  \includegraphics[width=\textwidth]{images/gufi_query_processdir.png}
  \caption{Workflow of \processdir}
\end{figure}

\subsection{\querydb macro}
\querydb is the macro used to execute SQL statements and handle errors.

\begin{figure} [H]
  \centering
  \includegraphics[width=\textwidth]{images/querydb_macro.png}
  \caption{\querydb macro workflow}
\end{figure}
